\subsection{Backend Source Structure}
All the backend code is stored in the ITD/CLupServer folder.
\begin{lstlisting}
    CLupServer\

    |-- docker-compose.yml
    |-- docker-compose-testing.yml
    |-- build.sh
    |-- test.sh
    |-- clup-server\
    |-- data\
    |-- db_config\
\end{lstlisting}

\begin{itemize}
    \item \textbf{docker-compose.yml}: docker-compose configuration file for setting up the production server. Docker-compose will read this file and build/set-up the containers accordingly.
    \item \textbf{docker-compose-testing.yml}: docker-compose configuration file for running the tests. Docker-compose will read this file and build/set-up the containers accordingly, containerizing all components but not the flask server.
    \item \textbf{build.sh}: A shell script that builds (if not already done) and starts the docker containers of the production server.
    \item \textbf{test.sh}: A shell script that starts the docker containers and runs pytest.
    \item \textbf{data}: Contains nginx configuration files.
    \item \textbf{db-config} Contains a SQL script that runs when the PostgreSQL is created. This script creates the databases (clup, clup-testing)
\end{itemize}

\begin{lstlisting}
    CLupServer\clup-server\

    |-- CLup.py
    |-- wsgi.py
    |-- config.py
    |-- data.json
    |-- docker-entrypoint.sh
    |-- Dockerfile
    |-- poetry.lock
    |-- pyproject.toml
    |-- README.rst
    |-- requirements.txt
    |-- clup_server\
    |-- tests\


\end{lstlisting}

\begin{itemize}
    \item \textbf{CLup.py}: The startup script for the flask application. This script is used to run the project when flask is not containerized. Some optional flag could be passed to this script (preceded by two dashes):
          \begin{itemize}
              \item dev:  For setting up Flask in development mode
              \item drop: For dropping the database content on startup
              \item populate: For populating the CLupUser and Store tables from the sample data stored in data.json
          \end{itemize}
    \item \textbf{wsgi.py}: The startup script for the application when it is run from a production WSGI server (i.e. Gunicorn). This is startup script is used when the app is containerized.
    \item \textbf{config.py}: Stores the configuration variables for flask for Development mode and for Production mode. The JWT secret key for the Production server is taken from the environment variables to avoid publishing the secret in the repository. The secret should be set to a random string (using export) before starting the server (This is automatically done from the build.sh script).
    \item \textbf{data.json}: Contains fake Stores and Users data to populate the database.
    \item \textbf{docker-entrypoint.sh}: A shellscript executed from Docker when it starts the application container. This script starts gunicorn.
    \item \textbf{Dockerfile}: Contains the instructions to allow Docker to build the application container.
    \item \textbf{pyproject.toml}: Configuration file for poetry. Contains the list of all the library dependencies needed to run the project.
    \item \textbf{requirements.txt}: Configuration file for pip. Used from the application container to retrieve all the project dependencies.
    \item \textbf{clup-server}: Contains the application python package
    \item \textbf{tests}: Contains the integration tests
\end{itemize}

\begin{lstlisting}
    CLupServer\clup-server\

    |-- __init__.py
    |-- models.py
    |-- schemas.py
    |-- routes.py
    |-- orm.py
    |-- auth_manager.py
    |-- information_provider.py
    |-- queue_manager.py
    |-- ticket_manager.py

\end{lstlisting}

\begin{itemize}
    \item \textbf{\_\_init\_\_.py} is the package initialization file, executed by Python when the clup-server package is imported from another source file. Here is implemented the application factory method createApp(). This method allows multiple instances with different configurations to be created. The createApp methods will import all the other modules in the package and then starts to inizialize all the needed resources. For example it will start the database connection, configure the API routes\ldots
    \item \textbf{models.py} contains all the models class declarations. Each model class is mapped to a database table, and SQLAlchemy provides the translation to a SQL statement for every operation made on a instance of one of the model classes. Each class contains also methods to decouple the data layer management code from the business code executed at each API call
    \item \textbf{routes.py} registers on start up all the API routes exposed to be accessed by the application
    \item \textbf{auth-manager.py} contains the APIs and the business logic for authenticating customers and operators.
    \item \textbf{information-provider.py} contains the APIs that provide information about the stores.
    \item \textbf{queue-manager.py} contains the APIs and the business logic used by the store operators(or automated control systems) to get the status of the queue and manage it
    \item \textbf{ticket-manager.py} contains the APIs to create, view and cancel tickets.
\end{itemize}


\subsection{Frontend Source Structure}

All the frontend code is stored in the ITD/clup\_application folder.

\begin{lstlisting}
    clup_application\

    |-- android\
    |-- assets\
    |-- fonts\
    |-- ios\
    |-- lib\
    |-- test\
    |-- web\
    |-- .gitignore
    |-- .metadata
    |-- pubspec.lock
    |-- pubspec.yaml
    |-- README.md 

\end{lstlisting}

\begin{itemize}
    \item \textbf{android/}: folder containing mostly configuration files to correctly build the android application. Most of these are autogenerated files, with little tweaks, for example to setup Android permissions.
    \item \textbf{assets/}: here are stored all the needed assets files (in this case only the CLup logo).
    \item \textbf{fonts/}: contains the collection of fonts used in the application.
    \item \textbf{ios/}: configurations to build the IOs application. has not been tweaked since the team had hardware limitations and could not test IOs builds.
    \item \textbf{lib/}: contains the whole Flutter code of the application.
    \item \textbf{web/}: contains configuration files for the webapp build.
    \item \textbf{.gitignore}: autogenerated by Flutter, avoids pushing build and other local configuration files to the git repo
    \item \textbf{.metadata}: file used by Flutter to track the properties of the Flutter project. Autogenerated and should not be manually edited.
    \item \textbf{pubspec.lock}: autogenerated, used by Flutter to store information about dependencies
    \item \textbf{pubspec.yaml}: main configuration file for the dependencies of the Flutter project.
    \item \textbf{README.md}: text to be displayed in the git repo to provide info about the folder.
\end{itemize}



\begin{lstlisting}
    lib\

    |-- api\
    |------- authentication.dart
    |------- information_provider.dart
    |------- operator_utils.dart
    |------- ticket_handler.dart
    |-- conditional_deps\
    |------- key_finder_stub.dart
    |------- keyfinder_interface.dart
    |------- mobile_keyfinder.dart
    |------- web_keyfinder.dart
    |-- gps\
    |------- gps_component.dart
    |-- pages\
    |------- login_page.dart
    |------- map_page.dart
    |------- operator_page.dart
    |------- signup_confirm_page.dart
    |------- signup_page.dart
    |------- store_view_page.dart
    |------- ticket_page.dart
    |-- configs.dart
    |-- generated_plugin_registrant.dart
    |-- main.dart

\end{lstlisting}

\begin{itemize}
    \item \textbf{api/}: this package contains all the calls to the CLup API tranformed in simple utils as in a Facade pattern, for easy access throughout the whole application.
    \item \textbf{conditional\_deps/}: this packages contains different implementations of local storage management, to avoid a problem when working with crossplatform applications in Flutter; it is not possible to ship an application with web libraries as dependencies, so the application instantiates different behaviours to correctly match the underling environment.
    \item \textbf{gps/}: in this packages is present the code that is used to retrieve the user location. The method determinePosition() controls them location permissions to use the gps sensor, and in case of no sensor or no permissions, makes a call to an external API to geolocalize using the ip address.
    \item \textbf{pages/}: this package contains all the pages of the application, and the majority of the code takes care of the layouts, the visual componenents, and the buttons. There is very little application logic, since all the work is done on the backend.
    \item \textbf{configs.dart}: contains global variables like URLS and custom colors, to be easily accessed throughout the whole application.
    \item \textbf{generated\_plugin\_registrant.dart}: autogenerated by google\_maps\_flutter\_web, not to be edited manually.
    \item \textbf{main.dart}: code to create and launch the application, here is defined the theme of the CLup app, the hierarchy structure of the pages, and utils function to write/read from local storage (implementing the correct keyfinder from the conditional dependencies).
\end{itemize}