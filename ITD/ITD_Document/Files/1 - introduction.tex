\subsection{Scope}
This document contains all information regarding the development and the testing of the CLup prototype application and system.

The ITD is a follow-up to the \textbf{RASD} (Requirement Analysis and Specification Document) and the \textbf{DD} (Design Document) which define in detail the system as a whole, all the planned features and the design choices.

\subsection{Document Structure}

\textbf{Chapter 1} explains the purpose of this document and the relation between this Implementation and Testing Document and the RASD and the DD.
Acronyms and Definitions used through the whole document are listed and explained.
It provides the history of the document revisions and this very description of all the chapters, together with the documents used as a reference.

\medskip

\textbf{Chapter 2} lists all the defined Requirements in the other documents and provides details about the fullfilling of those requirements. Considering that presented application is a prototype, not every functionality has been implemented.

Partially implemented features are also discussed in detail.

\medskip

\textbf{Chapter 3} presents all the adopted Frameworks both on the backend and on the frontend. The implementation is put into perspective of the defined and analyzed system structure showed in the Design Document. Every choice about existing software is provided in this chapter.

\medskip


\textbf{Chapter 4} describes the structure of the source code for both the frontend and backend. Useless information about autogenerated code/configs is omitted if not needed to explain implementation aspects.

\medskip


\textbf{Chapter 5} analyzes all aspects regarding the testing of the system. Adopted methodologies and achieved results are all presented in this chapter, together with the test coverage.

\medskip

\textbf{Chapter 6} provides insights on how the system has been deployed for manual testing, how to run locally the server or the client application, and which scripts are provided to do so.

\medskip

\textbf{Chapter 7} shows the time spent from each member of the team for writing this document.

\clearpage
\subsection{Definitions, Acronyms, Abbreviations}

\subsubsection{Acronyms}

\begin{itemize}
    \item \textbf{S2B}: Software To Be
    \item \textbf{RASD}: Requirement Analysis and Specifications Documents
    \item \textbf{REST}: REpresentational State Transfer
    \item \textbf{API}: Application Programming Interface
    \item \textbf{UX}: User Experience
    \item \textbf{UI}: User Interface
    \item \textbf{SSO}: Single sign-on
    \item \textbf{QR code}: Quick Response code
    \item \textbf{OS}: Operating System
    \item \textbf{RAM}: Random Access Memory
    \item \textbf{LAN}: Local Area Network
    \item \textbf{GPS}: Global Positioning System
    \item \textbf{GB}: GigaByte
    \item \textbf{TCP/IP}: Transmission Control Protocol/Internet Protocol
    \item \textbf{HTTPS}: Hypertext Transfer Protocol Secure
    \item \textbf{IoT}: Internet of Things
    \item \textbf{MQTT}: Message Queuing Telemetry Transport
    \item \textbf{RAID}: Redundant Array of Independent Disks
    \item \textbf{UML}: Unified Modeling Language
    \item \textbf{TDD}: Test-driven development
\end{itemize}

\vfill
\pagebreak

\subsubsection{Definitions}

\begin{itemize}
    \item \textbf{Access controller}: a subsystem that permits the entrance of customers into the store. It can be a device like a smart turnstile that reads customers tickets or just a person of the store staff manually scanning tickets.
    \item \textbf{Business account}: a CLup account that is destined to store managers or operators and therefore the 'business' side of CLup
    \item \textbf{In-Site ticket}: a ticket that is taken by a customer near the store. It can be both a virtual paperless ti an emitter near the store premises
    \item \textbf{Virtual ticket}: a ticket issued through the CLup application
    \item \textbf{Physical/Paper ticket}: a printed physical ticket, emitted by a printer near the store premises
    \item \textbf{Valid Ticket} (at time X): a ticket that has a code recognized by the CLup system and valid for the specified time
    \item \textbf{Time slot}: a time delta that is associated with a number of bookable tickets (which varies and is customizable from store to store)
    \item \textbf{People-Counting System}: a subsystem that permits the counting of the number of people inside the store. It can comprehend a device like a proximity sensor or a smart turnstile, or it can be a person of the store staff manually counting people.
    \item \textbf{Customer Application}: the CLup mobile application destined to customers that want to shop inside stores adopting CLup
    \item \textbf{Operator Application}: the CLup mobile application destined to store staff to monitor entrances and statistics
    \item \textbf{Store Main System}: the store main server that communicates directly with CLup servers. All store subsystems and smart devices should communicate with it through an Intranet
    \item \textbf{Geocoding API}: Geocoding converts addresses into geographic coordinates to be placed on a map. A Geocoding API allows the use of their services to permit translation between textual addresses and Latitude/Longitude coordinates
    \item \textbf{Map API}: An external services that provides operations of geographics maps and the download of map information, usually of the places in proximity of given geographics coordinates
    \item \textbf{Hashed Password}: When a password has been “hashed” it means it has been turned into a scrambled representation of itself. A user's password is taken and – using a key known to the site – the hash value is derived from the combination of both the password and the key, using a set algorithm.
    \item \textbf{Time to market}: is the length of time it takes from a product being conceived until its being available for sale.
    \item \textbf{Alpha Test}: a trial of machinery, software, or other products carried out by a developer before a product is made available for beta testing.
    \item \textbf{Beta Test}: a trial of machinery, software or other products in the final stages of development, carried out by a party unconnected with the development process.
    \item \textbf{DevOps}: is a set of practices that combines software development (Dev) and IT operations (Ops). It aims to shorten the systems development life cycle and provide continuous delivery with high software quality.
    \item \textbf{Quality Assessment}: is the data collection and analysis through which the degree of conformity to predetermined standards and criteria are exemplified.
\end{itemize}

\subsection{Revision History}

\begin{itemize}
    \item 1.0: 10 January 2021 - First Release
\end{itemize}

\subsection{Reference Documents}

\begin{itemize}
    \item R\&DD Assignment A.Y. 2020-2021 - Elisabetta di Nitto, Matteo Giovanni Rossi
    \item Software Engineering II slides and material - Elisabetta di Nitto, Matteo Giovanni Rossi
    \item CLup RASD - Davide Luca Merli, Dario Passarello
    \item \href{https://material.io/design}{Material Design Guidelines}
    \item \href{https://www.conventionalcommits.org/en/v1.0.0/}{Convetional Commits Guidelines}
    \item \href{https://git-scm.com/}{Git - versioning system}
\end{itemize}

\subsection{Used Tools}
\begin{itemize}
    \item Umlet - for UML diagrams
    \item Github - for code hosting and version control
    \item \LaTeX \space  - to write this entire document
    \item Visual Studio Code + Latex Workshop - as a \LaTeX \space environment
    \item Visual Studio Code + Flutter \& Dart plugins - as a Flutter environment
    \item Postman - to test the REST API
    \item pgAdmin - to manage the postgres test dataset
    \item Amazon Web Services - to deploy the backend
    \item Github Pages - to deploy the web application
\end{itemize}

\clearpage