\subsection{Backend}
The backend server has been deployed with Docker containers on AWS.



\subsubsection{Starting the server}
An instance of the server has already been deployed online using AWS, and the API is available at https://clup.waifocus.com so it's not needed to run it locally.
If you want it is still possible to run the server locally on Linux or on Windows Subsystem of Linux (WSL)
\textbf{Installation steps}
\begin{enumerate}
    \item Install \href{https://docs.docker.com/get-docker/}{Docker}
    \item Install \href{https://docs.docker.com/compose/install/}{Docker Compose}
    \item Check if Docker service is running. If not start it with the command
    \begin{lstlisting}[language=bash]
    sudo systemctl start docker
    \end{lstlisting}
    \item  Open build.sh and change the JWT\_SECRET\_KEY string to another value. When running the application in a production environment keep this value secret.
    \item Run the script with 
    \begin{lstlisting}[language=bash]
    ./build.sh
    \end{lstlisting}
    this script will create the containers invoking docker-compose. If you didn't add your user in the docker group 
    you have to run the script with sudo
    \item You can do requests the address localhost:8000 
\end{enumerate}

\subsubsection{Running tests}
For running pytest a startup script that doesn't containerize the application is available.
\textbf{Installation steps}
\begin{enumerate}
    \item Install \href{https://docs.docker.com/get-docker/}{Docker}
    \item Install \href{https://docs.docker.com/compose/install/}{Docker Compose}
    \item Check if Docker service is running. If not start it with the command
    \begin{lstlisting}[language=bash]
    sudo systemctl start docker
    \end{lstlisting}
    \item Install \href{https://python-poetry.org/}{Poetry}
    \item Open a linux shell and go in CLupServer/clup-server 
    \item Run the command 
    \begin{lstlisting}[language=bash]
    poetry install
    \end{lstlisting}
    \item Run the test script with 
    \begin{lstlisting}[language=bash]
    ./test.sh
    \end{lstlisting}
    The script sets up all the containers and the flask application in testing mode and the it runs the Pytest.
    If you didn't add your user in the docker group 
    you have to run the script with sudo
    \item (Optional) If you want to start the application out of the container run the command
    \begin{lstlisting}[language=bash]
    poetry run python CLup.py
    \end{lstlisting}
\end{enumerate}

\subsection{Frontend}

\subsubsection{Flutter Mobile Application}
The application has been developed and tested only on Android, due to limitation in the available hardware.

To install the application we provide two~.apk packages:
\begin{itemize}
    \item RELEASE Version: A smaller package containing the prototype as it would be released in a production environment, downloadable here
    \item DEBUG Version: A bigger package containing the debug version of the prototype, which is useful for checking potential error messages, full content of API calls, etc. Downloadable here
\end{itemize}

\subsubsection{Flutter Web Application}
A release version of the web application has been deployed using Github Pages, and it is available at https://clup-mp.github.io, without the need to install anything.

We also provide a package containing all files to deploy the web application, downloadable here

To test locally the web application it is possible to setup a simple web server, for example by using python:
\begin{lstlisting}[language=bash]
    python3 -m http.server
\end{lstlisting}
Then opening a browser on http://localhost:8000

