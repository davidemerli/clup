\subsection{Purpose}

\subsubsection{Problem Analysis}

Due to the recent coronavirus global pandemic, many of the human activities have been drastically affected and restricted by the need to contain the virus diffusion. Norms and regulations may vary from contry to country, but almost everywhere the main focus is to mantain social distancing to avoid diffusion from one person to another.
One of the most difficult activity to fullfil (yet absolutely essential) is grocery shopping.
Stores are forced to restrain access to avoid too many people inside the building, and this produces endless lines out of the stores. This can both increase the danger for people waiting for their turn and force the shops to regulate customers even outside the structure.
CLup aims to reduce heavily the issues involving customer queues outside of stores by permitting clients to keep track of their position in the store queue or book in a visit in advance with an easy to use application.
On the other hand, CLup also will provide a simple (but very customizable and scalable)system for the stores to integrate the queue monitoring tools into their everyday workflow, without the absolute need to buy expensive hardware.
Moreover, CLup will take in consideration the fact that not every customer has the same familiarity with today’s technology. In addiction to a straightforward interface and UX, the system will permit the stores to hand out tickets on the spot. It should also be possible to improve the overall experience in line even to the customers that do not use the CLup application, thanks to the monitoring tools that will be made available.
CLup has the goal to ease the struggles for customers and store workers and save a lot of time that would have been lost waiting for hours in queues. This is an important improvement, even in times not as trying as the current coronavirus emergency. A well integrated system taking advantage of the CLup service can even boost sales: clients are more willing to go shopping if they know they won’t be loosing time; store owners will also have the possibility to check and profit from statistics about user entrances and average shopping time
\medskip

Stores are forced to restrain access to avoid too many people inside the building,
and this produce endless lines out of the stores. This can both increase the danger for people waiting for their turn and force the shops to regulate customers even outside the structure.

\medskip

CLup aims to reduce heavily the issues involving customer queues outside of stores by permitting clients to keep track of their position in the store queue or book in a visit in advance with an easy to use application.

\medskip

On the other hand, CLup also will provide a simple (but very customizable and scalable) system for the stores to integrate the queue monitoring tools into their everyday workflow, without the absolute need to buy expensive hardware.

\medskip

Moreover, CLup will take in consideration the fact that not every customer has the same familiarity with today's technology.
In addiction to a straightforward interface and UX, the system will permit the stores to hand out tickets on the spot. It should be also possible to improve the overall experience in line even to the customers that do not use the CLup application, thanks to the monitoring tools that will be made available.
\medskip
CLup has the goal to ease the struggles for customers and store workers and save a lot of time that would have been lost waiting for hours in queues. This is an important improvement, even in times not as trying as the current coronavirus emergency.
A well integrated system taking advantage of the CLup service can even boost sales: clients are more willing to go shopping if they know they won't be loosing time; store owners will also have the possibility to check and profit from statistics about user entrances and average shopping time.

\subsubsection{Document Purpose}

The purpose of this document is to analyze the problem taking in consideration the real needs of the customers and shop workers.
This RASD describes in detail the functional and non-functional requirements of the S2B and includes exhaustive descriptions about typical use cases from the actors that will take part of the system.
Hence, this document is addressed to the developers of the S2B as well as the companies that want to integrate CLup services into their workflow.

\vfill

\pagebreak

\subsubsection{Goals}
\rowcolors{2}{gray!25}{white}
\renewcommand{\arraystretch}{1.4}
\captionof{table}{Goals table}
\begin{tabular}{C{1cm}L{14.1cm}}
      \rowcolor{gray!50}
      Label & Goal Description                                                                                                                                                               \\
      G1    & Avoid exceeding the maximum number of customer inside the store in each store                                                                                            \\
      G2    & Reduce the number of customer waiting physically in line in front of the store entrance                                                                                \\
      G3    & Try to distribute people uniformly inside the store to ease maintaining social distancing                                                                                      \\
      G4    & Allow customers to book for a visit to the store at a desired time and day                                   \\
      G5    & Let every customer have the possibility to access the store regardless of the technology available to them                                                                      \\
      G6    & Give stores adopting CLup access to anonymous statistics regarding the people coming to the store                                                                               \\
      G7    & Provide a simple and user-friendly interface to book tickets                                                                                                                   \\
      G8    & Provide an interface for the access controller to check tickets and monitor the occupancy                                                                                      \\
      G9    & Provide an estimation of the waiting time to every customer that waits in line (physical or virtual)  \\
      G10   & Notify a customer when it's time to approach the store entrance      \\
\end{tabular}

\subsection{Scope}
CLup positions itself as an intermediary between stores and customers. Clients can book entrances and retrieve tickets through the application, and the stores communicate with the CLup backend to update entrances, leavings and the capacity of the building.

\medskip

The mobile application can be used by customers to monitor their time in queue, get notifications when they should approach the entrance and get a time estimation before it's their turn to enter; it can also be used by store employees to manually update the store live information.

\medskip

Furthermore, CLup will provide a simple but powerful REST API that can be easily exploited to automate completely the process of updating live store information.

\vfill

\pagebreak

\subsection{Definitions, Acronyms, Abbreviations}

\subsubsection{Acronyms}

\begin{itemize}
      \item \textbf{S2B}: Software To Be
      \item \textbf{RASD}: Requirement Analysis and Specifications Documents
      \item \textbf{REST}: REpresentational State Transfer
      \item \textbf{API}: Application Programming Interface
      \item \textbf{UX}: User Experience
      \item \textbf{UI}: User Interface
      \item \textbf{SSO}: Single sign-on
      \item \textbf{QR code}: Quick Response code
      \item \textbf{OS}: Operating System
      \item \textbf{RAM}: Random Access Memory
      \item \textbf{LAN}: Local Area Network
      \item \textbf{GPS}: Global Positioning System
      \item \textbf{GB}: GigaByte
      \item \textbf{TCP/IP}: Transmission Control Protocol/Internet Protocol
      \item \textbf{HTTPS}: Hypertext Transfer Protocol Secure
      \item \textbf{IoT}: Internet of Things
      \item \textbf{MQTT}: Message Queuing Telemetry Transport
      \item \textbf{RAID}: Redundant Array of Independent Disks


\end{itemize}

\subsubsection{Definitions}
\begin{itemize}
      \item \textbf{Access controller}: a subsystem that permits the entrance of customers into the store. It can be a device like a smart turnstile that reads customers tickets or just a person of the store staff manually scanning tickets.
      \item \textbf{Business account}: a CLup account that is destined to store managers or operators and therefore the 'business' side of CLup
      \item \textbf{In-Site ticket}: a ticket that is taken by a customer near the store. It can be both a virtual paperless ticket or a ticket printer by an emitter near the store premises
      \item \textbf{Virtual ticket}: a ticket issued through the CLup application
      \item \textbf{Physical/Paper ticket}: a printed physical ticket, emitted by a printer near the store premises
      \item \textbf{Valid Ticket} (at time X): a ticket that has a code recognized by the CLup system and valid for the specified time
      \item \textbf{Time slot}: a time delta that is associated with a number of bookable tickets (which varies and is customizable from store to store)
      \item \textbf{People-Counting System}: a subsystem that permits the counting of the number of people inside the store. It can be a device like a proximity sensor or a smart turnstile, or it can be a person of the store staff manually counting people.
      \item \textbf{Customer Application}: the CLup mobile application destined to customers that want to shop inside stores adopting CLup
      \item \textbf{Operator Application}: the CLup mobile application destined to store staff to monitor entrances and statistics
      \item \textbf{Store Main System}: the store main server that communicates directly with CLup servers. All store subsystems and smart devices should communicate with it through an Intranet
\end{itemize}

\subsection{Reference Documents}

\begin{itemize}
      \item The World and the Machine - Michael Jackson - ICSE95: 17th International Conference on Software Engineering Seattle Washington USA April, 1995
      \item R\&DD Assignment A.Y. 2020-2021 - Elisabetta di Nitto, Matteo Giovanni Rossi
      \item Software Engineering II slides and material - Elisabetta di Nitto, Matteo Giovanni Rossi
      \item ISO/IEC/IEEE 29148:2018 Systems and software engineering - Life cycle processes - Requirements engineering
\end{itemize}

\subsection{Document structure}

\textbf{Chapter 1} introduces the problems that the S2B will resolve framing them into the current world situation. It describes purpose of the project and the goals that the software will need to accomplish. Moreover, it reports reference documents and describes the lexicon used in the whole document (i.e. Definitions, Acronyms and Abbreviations).

\textbf{Chapter 2} lays out an in-depth analysis of the functionalities that will be provided by the S2B starting from the examination of real world phenomena, user characteristics and details about demographics differences on the customers side.

In detail are shown:
\begin{itemize}
      \item specific description of likely scenarios of customers everyday life and their relation with CLup
      \item functional and non-functional requirements of the S2B
      \item domain assumptions about the environment in which the system behaves
      \item a formal representation of the whole system
\end{itemize}

\textbf{Chapter 3} examines specific requirements considered together with current technologies, hardware, systems and existing interfaces. It provides insights about CLup relationship with the external world from a technology point of view. It lists the constraints that needs to be respected, and delineates system attributes like Reliability, Availability, Security, Maintainability and Portability

Common use cases are meticulously described, divided into smaller actions and highlighting the actors taking place in them. Sequence Diagrams are provided for every use case in order to further explain and visualize them.

Requirements are mapped with the goals they accomplish, together with the domain assumptions they are related to. This improves the understanding of how the work can later be divided and parallelized, and will define the development process.


\textbf{Chapter 4} provides an abstract static Alloy model for the S2B. In this section the goals of the modelling activity are explained. The commented Alloy code is provided to the reader, this code is further explained with some instances generated from the alloy engine.

\textbf{Chapter 5} shows the time spent from each member of the team for writing this document.