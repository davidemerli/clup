In this section are discussed the details about the implementation of the system, showing a possible (but not unique) implementation plan, an unit and integration testing strategy and some useful DevOps practices that could ease and speedup all the system development process


\subsection{Implementation Plan}
As shown in the Figure~\ref{fig:UML_comp_general} there are three macro component to implement from scratch. The other components either are up and running (i.e.~a Map Service) or are already developed and the only step to do is set up an instance of them (i.e. Databases, Physical Components\ldots). 
The only thing the system needs to do is to interface with the in a correct way.

The three macro-components to implement are the CLup User Application, the CLup Server and the Store System. Because all the common interfaces were previously specified (See section]\ref{sect:requirement_traceability}) each of these three components could be implemented independently. This allows three different teams to work in parallel in the implementation of each macro component speeding up the development process.

For each macro-component a possible implementation will be defined keeping in consideration the testing aspect. For testing a component, or a set of components a \textit{driver} simulating another component using the interface for the tested component has to be coded. Also when the tested component uses another component's interface that's not yet implemented a \textit{stub} simulating that component has to be written. 

Writing \textit{drivers} and \textit{stubs} could take some time and reducing their count allows to reduce the overhead of the testing.

Another way to reduce 

\subsubsection{CLup User Application}
For this macro-component the development should start from the controller modules. 